% Options for packages loaded elsewhere
\PassOptionsToPackage{unicode}{hyperref}
\PassOptionsToPackage{hyphens}{url}
\PassOptionsToPackage{dvipsnames,svgnames,x11names}{xcolor}
%
\documentclass[
]{book}

\usepackage{amsmath,amssymb}
\usepackage{iftex}
\ifPDFTeX
  \usepackage[T1]{fontenc}
  \usepackage[utf8]{inputenc}
  \usepackage{textcomp} % provide euro and other symbols
\else % if luatex or xetex
  \usepackage{unicode-math}
  \defaultfontfeatures{Scale=MatchLowercase}
  \defaultfontfeatures[\rmfamily]{Ligatures=TeX,Scale=1}
\fi
\usepackage{lmodern}
\ifPDFTeX\else  
    % xetex/luatex font selection
\fi
% Use upquote if available, for straight quotes in verbatim environments
\IfFileExists{upquote.sty}{\usepackage{upquote}}{}
\IfFileExists{microtype.sty}{% use microtype if available
  \usepackage[]{microtype}
  \UseMicrotypeSet[protrusion]{basicmath} % disable protrusion for tt fonts
}{}
\makeatletter
\@ifundefined{KOMAClassName}{% if non-KOMA class
  \IfFileExists{parskip.sty}{%
    \usepackage{parskip}
  }{% else
    \setlength{\parindent}{0pt}
    \setlength{\parskip}{6pt plus 2pt minus 1pt}}
}{% if KOMA class
  \KOMAoptions{parskip=half}}
\makeatother
\usepackage{xcolor}
\setlength{\emergencystretch}{3em} % prevent overfull lines
\setcounter{secnumdepth}{-\maxdimen} % remove section numbering
% Make \paragraph and \subparagraph free-standing
\ifx\paragraph\undefined\else
  \let\oldparagraph\paragraph
  \renewcommand{\paragraph}[1]{\oldparagraph{#1}\mbox{}}
\fi
\ifx\subparagraph\undefined\else
  \let\oldsubparagraph\subparagraph
  \renewcommand{\subparagraph}[1]{\oldsubparagraph{#1}\mbox{}}
\fi


\providecommand{\tightlist}{%
  \setlength{\itemsep}{0pt}\setlength{\parskip}{0pt}}\usepackage{longtable,booktabs,array}
\usepackage{calc} % for calculating minipage widths
% Correct order of tables after \paragraph or \subparagraph
\usepackage{etoolbox}
\makeatletter
\patchcmd\longtable{\par}{\if@noskipsec\mbox{}\fi\par}{}{}
\makeatother
% Allow footnotes in longtable head/foot
\IfFileExists{footnotehyper.sty}{\usepackage{footnotehyper}}{\usepackage{footnote}}
\makesavenoteenv{longtable}
\usepackage{graphicx}
\makeatletter
\def\maxwidth{\ifdim\Gin@nat@width>\linewidth\linewidth\else\Gin@nat@width\fi}
\def\maxheight{\ifdim\Gin@nat@height>\textheight\textheight\else\Gin@nat@height\fi}
\makeatother
% Scale images if necessary, so that they will not overflow the page
% margins by default, and it is still possible to overwrite the defaults
% using explicit options in \includegraphics[width, height, ...]{}
\setkeys{Gin}{width=\maxwidth,height=\maxheight,keepaspectratio}
% Set default figure placement to htbp
\makeatletter
\def\fps@figure{htbp}
\makeatother

\makeatletter
\makeatother
\makeatletter
\makeatother
\makeatletter
\@ifpackageloaded{caption}{}{\usepackage{caption}}
\AtBeginDocument{%
\ifdefined\contentsname
  \renewcommand*\contentsname{Table of contents}
\else
  \newcommand\contentsname{Table of contents}
\fi
\ifdefined\listfigurename
  \renewcommand*\listfigurename{List of Figures}
\else
  \newcommand\listfigurename{List of Figures}
\fi
\ifdefined\listtablename
  \renewcommand*\listtablename{List of Tables}
\else
  \newcommand\listtablename{List of Tables}
\fi
\ifdefined\figurename
  \renewcommand*\figurename{Figure}
\else
  \newcommand\figurename{Figure}
\fi
\ifdefined\tablename
  \renewcommand*\tablename{Table}
\else
  \newcommand\tablename{Table}
\fi
}
\@ifpackageloaded{float}{}{\usepackage{float}}
\floatstyle{ruled}
\@ifundefined{c@chapter}{\newfloat{codelisting}{h}{lop}}{\newfloat{codelisting}{h}{lop}[chapter]}
\floatname{codelisting}{Listing}
\newcommand*\listoflistings{\listof{codelisting}{List of Listings}}
\makeatother
\makeatletter
\@ifpackageloaded{caption}{}{\usepackage{caption}}
\@ifpackageloaded{subcaption}{}{\usepackage{subcaption}}
\makeatother
\makeatletter
\@ifpackageloaded{tcolorbox}{}{\usepackage[skins,breakable]{tcolorbox}}
\makeatother
\makeatletter
\@ifundefined{shadecolor}{\definecolor{shadecolor}{rgb}{.97, .97, .97}}
\makeatother
\makeatletter
\makeatother
\makeatletter
\makeatother
\ifLuaTeX
  \usepackage{selnolig}  % disable illegal ligatures
\fi
\IfFileExists{bookmark.sty}{\usepackage{bookmark}}{\usepackage{hyperref}}
\IfFileExists{xurl.sty}{\usepackage{xurl}}{} % add URL line breaks if available
\urlstyle{same} % disable monospaced font for URLs
\hypersetup{
  pdftitle={CM 1015 Computational Mathematics},
  pdfauthor={Kevin Juandi},
  colorlinks=true,
  linkcolor={Maroon},
  filecolor={Maroon},
  citecolor={Blue},
  urlcolor={Blue},
  pdfcreator={LaTeX via pandoc}}

\title{CM 1015 Computational Mathematics}
\author{Kevin Juandi}
\date{2023-06-27}

\begin{document}
\frontmatter
\maketitle
\ifdefined\Shaded\renewenvironment{Shaded}{\begin{tcolorbox}[interior hidden, sharp corners, enhanced, borderline west={3pt}{0pt}{shadecolor}, frame hidden, boxrule=0pt, breakable]}{\end{tcolorbox}}\fi

\mainmatter
\hypertarget{preface}{%
\chapter*{Preface}\label{preface}}
\addcontentsline{toc}{chapter}{Preface}

I wrote this note after finishing the course, so the content might not
reflect the current version of the course. It is mostly based on my
handwritten personal notes. I personally feel this course should be
called ``Foundation Mathematics'' instead of ``Computational
Mathematics'' because of the lack of ``Numerical Methods'' and probably
some other things people more familiar with the topic would say. If you
spot any error please don't hesitate to contact me via slack or mail me.

\hypertarget{number-bases-conversion-and-operations}{%
\chapter{Number Bases, Conversion and
Operations}\label{number-bases-conversion-and-operations}}

\newcommand*{\carry}[1][1]{\overset{#1}}

Reading Materials: \newline Croft, A. and R. Davidson \emph{Foundation
maths.} (Harlow: Pearson, 2016) 6th edition. \textbf{Chapter 14 Number
Bases}

\hypertarget{number-bases}{%
\section{Number Bases}\label{number-bases}}

\textbf{Decimal System}

The numbers that we commonly used are based on 10.\newline For example:
\begin{equation} \label{eq1}
\begin{split}
253 & = 200 + 50 + 3 \\
& = 2(100) + 5(10) + 3 \\
& = 2(10^2)+ 5(10^1) + 3(10^0)
\end{split}
\end{equation}

\textbf{Binary System}

\normalsize A binary system uses base 2, it only consist of 2 digits, 0
and 1.\newline Numbers in base 2 are called binary digits or simply
bits.\newline Consider the binary number \(110101_2\). As the base is 2,
this means that power of 2 essentially replace powers of 10. Let us
convert it to base 10. \begin{equation} \label{eq2}
\begin{split}
110101_2 & = 1(2^5)+1(2^4)+0(2^3)+1(2^2)+0(2^1)+1(2^0) \\
& = 1(32)+1(16)+0(8)+1(4)+0(2)+1(1) \\
& = 32+16+4+1 \\
& = 53_{10}
\end{split}
\end{equation}

\textbf{Octal System}

\normalsize Octal numbers use 8 as a base. The eight digits used in the
octal system are 0, 1, 2, 3, 4, 5, 6 and 7. Octal numbers use powers of
8, just as decimal numbers use powers of 10 and binary numbers use
powers of 2. Example: \begin{equation} \label{eq3}
\begin{split}
325_8 & = 3(8^2)+2(8^1)+5(8^0) \\
& = 3(64)+2(8)+5(1) \\
& = 192+16+5 \\
& = 213_{10}
\end{split}
\end{equation}

\textbf{Hexadecimal System}

\normalsize Hexadecimal system use 16 as a base. The digits are 0, 1, 2,
3, 4, 5, 6, 7, 8, 9, A, B, C, D, E and F. Example:
\begin{equation} \label{eq4}
\begin{split}
93\text{A}_{16} & = 9(16^2)+3(16^1)+\text{A}(16^0) \\
& = 9(256)+3(16)+10(1) \\
& = 2304+48+10 \\
& = 2362_{10}
\end{split}
\end{equation}

\hypertarget{number-conversion}{%
\section{Number Conversion}\label{number-conversion}}

\hypertarget{converting-from-decimal-to-other-number-base}{%
\subsection{Converting from Decimal to Other Number
Base}\label{converting-from-decimal-to-other-number-base}}

\noindent The no-brainer way is to divide the number by the base, the
remainder would be the last digit of the new number base. Keep dividing
the quotient until it is smaller than the number base. Let us convert
\(253_{10}\) as example. \begin{equation} \label{eq5}
\begin{split}
2{\overline{\smash{\big)}\,253\phantom{)}}} & = 126 \text{ with remainder }1\\
2{\overline{\smash{\big)}\,126\phantom{)}}} & = 63 \text{ with remainder }0\\
2{\overline{\smash{\big)}\,63\phantom{)}}} & = 31 \text{ with remainder }1\\
2{\overline{\smash{\big)}\,31\phantom{)}}} & = 15 \text{ with remainder }1\\
2{\overline{\smash{\big)}\,15\phantom{)}}} & = 7 \text{ with remainder }1\\
2{\overline{\smash{\big)}\,7\phantom{)}}} & = 3 \text{ with remainder }1\\
2{\overline{\smash{\big)}\,3\phantom{)}}} & = 1 \text{ with remainder }1\\
2{\overline{\smash{\big)}\,1\phantom{)}}} & = 0 \text{ with remainder }1
\end{split}
\end{equation} Thus, \(253_{10}\) is \(11111101_2\) in binary. We could
do the same to other number bases. \newline Another method is by listing
the powers of the base, compare and subtract. Using the same number as
example.

\begin{longtable}[]{@{}ccc@{}}
\toprule\noalign{}
\(2^0 = 1\) & \(2^1 = 2\) & \(2^2 = 4\) \\
\midrule\noalign{}
\endhead
\bottomrule\noalign{}
\endlastfoot
\(2^3 = 8\) & \(2^4 = 16\) & \(2^5 = 32\) \\
\(2^6 = 64\) & \(2^7 = 128\) & \(2^8 = 256\) \\
\end{longtable}

From here we compare the number that we are going to convert with the
list: \begin{equation} \label{eq6}
\begin{split}
253 - 1(128) & = 125  \\
125 - 1(64) & = 61 \\
61 - 1(32) & = 29 \\ 
29 - 1(16) & = 13 \\
13 - 1(8) & = 5 \\
5 - 1(4) & = 1 \\
1 - 0(2) & = 1\\
1- 1(1) & = 0
\end{split}
\end{equation} Thus, \(253_{10}\) is \(11111101_2\) in binary. Like the
other method, we can also do this to convert to other number bases.

\hypertarget{conversion-with-binary-number}{%
\subsection{Conversion with Binary
Number}\label{conversion-with-binary-number}}

\noindent Converting binary numbers to Octal or Hexadecimal and vice
versa are very straightforward. It can be performed without converting
to Decimal first. Let us use number \(11100110_2\) as example:
\begin{equation} \label{eq7}
\begin{split}
\underbrace{11}_\text{3}\underbrace{100}_\text{4}\underbrace{110}_\text{6} & = 346_8\\
\underbrace{1110}_\text{E}\underbrace{0110}_\text{6} & = \text{E}6_{16}
\end{split}
\end{equation}

\subsection{Non-integer Number Conversion}

Converting non-integer number might look counterintuitive and
intimidating. It is actually rather simple. Let us convert
\(17.375_{10}\) to binary. {[}

\begin{split}
17.375 & = 10 + 7 + 0.3+0.07+0.005 \\
& = 1(10^1) + 7(10^0) + 3(10^{-1}) + 7(10^{-2})+5(10^{-3})
\end{split}

{]} Converting to binary, \(17_{10} = 10001_2\). But how about the
decimal point? We multiply them by two until we are left with whole
number \begin{equation} \label{eq8}
\begin{split}
0.375 \times 2  = 0.75 = 0+0.75 & \text{ we have 0 at power }-1\\
0.75 \times 2  = 1.5 = 1+0.5 & \text{ we have 1 at power }-2\\
0.5 \times 2  = 1.0 = 1 & \text{ we have 1 at power }-3
\end{split}
\end{equation} Thus, \(17.375_{10} = 10001.011_2\)\newline The reverse
is much simpler. \begin{equation} \label{eq9}
\begin{split}
1101.101_2 & = 1(2^3)+1(2^2)+0(2^1)+1(2^0)+1(2^{-1})+0(2^{-2})+1(2^{-3}) \\
& = 1(8)+1(4)+0(2)+1(1)+1(0.5)+0(0.25)+1(0.125) \\
& = 8 + 4 + 1+ 0.5+0.125 \\
& = 13.625_{10}
\end{split}
\end{equation}

\hypertarget{operations-with-binary-number}{%
\subsection{Operations with Binary
Number}\label{operations-with-binary-number}}

\textbf{Addition}

\normalsize Addition is rather straightforward, just like with decimal
numbers. Here we use \(11001001_2\) and \(11111111_2\) as example. In
decimal they are 201 and 255 respectively. \begin{equation}\label{eq10}
    \begin{array}{B3}
        \overset{1}0 & \overset{1}1\overset{1}1\overset{1}0\overset{1}0 & \overset{1}1\overset{1}0\overset{1}01 \\
        {} + 0 &                             1111 &                      1111\\ \hline
        1 &                             1100 &                      1000\\
    \end{array}
\end{equation} \begin{equation} \label{eq11}
\underbrace{11001001}_\text{201}+\underbrace{11111111}_\text{255} = \underbrace{1 1100 1000}_\text{456}
\end{equation}

\textbf{Subtraction}

\normalsize Likewise for subtraction, between \(111 0011_2\) and
\(101 0010_2\) \begin{equation}\label{eq12}
    \begin{array}{B3}
         1  1 1 &  0 0 11\\
        {} -                              101 &                      0010\\ 
        \hline
        010 &                      0001 \\
    \end{array}
\end{equation} \begin{equation} \label{eq13}
    \underbrace{1110011}_\text{115}-\underbrace{1010010}_\text{82} = \underbrace{100001}_\text{33}
\end{equation}

\textbf{Multiplication}

\normalsize Multiplication is essentially addition done multiple times.
Let us have multiplication of \(1100_2\) and \(1111_2\). I would skip
the carry so it wouldn't look too cramped. \begin{equation}\label{eq14}
    \begin{array}{c}
        \phantom{\times9999}1100\\
        \underline{\times\phantom{9999}1111}\\
        \phantom{\times9999}1100\\
        \phantom{\times999}1100\phantom9\\
        \phantom{\times99}1100\phantom{99}\\
        \underline{\phantom{\times9}1100\phantom{999}}\\
        \phantom\times10110100
    \end{array}
\end{equation} \begin{equation} \label{eq15}
    \underbrace{1100}_\text{12}\times\underbrace{1111}_\text{15} = \underbrace{10110100}_\text{180}
\end{equation}

\textbf{Division}

\normalsize Division is perhaps the one that feels the most unnatural
and most likely to cause mistakes. Let us do this with \(11100110_2\)
divided with \(110_2\) as example. \begin{equation}\label{eq16}
\begin{array}{r}
100110\phantom{)}   \\
110{\overline{\smash{\big)}\,11100110\phantom{)}}}\\
\underline{-~\phantom{(}110\phantom{99999)}}\\
\phantom{\times{99}}10\phantom{9999)}\\ 
\underline{-~\phantom{()}0\phantom{9999)}}\\ 
\phantom{\times{99}}100\phantom{999)}\\ 
\underline{-~\phantom{()}0\phantom{999)}}\\ 
\phantom{\times{99}}1001\phantom{99)}\\ 
\underline{-~\phantom{()}110\phantom{99)}}\\ 
\phantom{\times{999}}111\phantom{9)}\\ 
\underline{-~\phantom{()}110\phantom{9)}}\\
\phantom{\times{9999}}10\phantom{)}\\ 
\underline{-~\phantom{()}0\phantom{)}}\\  
10\phantom{)}
\end{array}
\end{equation}

\begin{equation} \label{eq17}
    \underbrace{11100110}_\text{230}\div\underbrace{110}_\text{12} = \underbrace{100110}_\text{38} \text{ with remainder }\underbrace{10}_\text{2}
\end{equation}


\backmatter

\end{document}
